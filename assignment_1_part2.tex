\documentclass[journal,12pt,twocolumn]{IEEEtran}

\usepackage{setspace}
\usepackage{gensymb}

\singlespacing


\usepackage[cmex10]{amsmath}

\usepackage{amsthm}

\usepackage{mathrsfs}
\usepackage{txfonts}
\usepackage{stfloats}
\usepackage{bm}
\usepackage{cite}
\usepackage{cases}
\usepackage{subfig}

\usepackage{longtable}
\usepackage{multirow}

\usepackage{enumitem}
\usepackage{mathtools}
\usepackage{steinmetz}
\usepackage{tikz}
\usepackage{circuitikz}
\usepackage{verbatim}
\usepackage{tfrupee}
\usepackage[breaklinks=true]{hyperref}
\usepackage{graphicx}
\usepackage{tkz-euclide}

\usetikzlibrary{calc,math}
\usepackage{listings}
    \usepackage{color}                                            %%
    \usepackage{array}                                            %%
    \usepackage{longtable}                                        %%
    \usepackage{calc}                                             %%
    \usepackage{multirow}                                         %%
    \usepackage{hhline}                                           %%
    \usepackage{ifthen}                                           %%
    \usepackage{lscape}     
\usepackage{multicol}
\usepackage{chngcntr}

\DeclareMathOperator*{\Res}{Res}

\renewcommand\thesection{\arabic{section}}
\renewcommand\thesubsection{\thesection.\arabic{subsection}}
\renewcommand\thesubsubsection{\thesubsection.\arabic{subsubsection}}

\renewcommand\thesectiondis{\arabic{section}}
\renewcommand\thesubsectiondis{\thesectiondis.\arabic{subsection}}
\renewcommand\thesubsubsectiondis{\thesubsectiondis.\arabic{subsubsection}}


\hyphenation{op-tical net-works semi-conduc-tor}
\def\inputGnumericTable{}                                 %%

\lstset{
%language=C,
frame=single, 
breaklines=true,
columns=fullflexible
}
\begin{document}


\newtheorem{theorem}{Theorem}[section]
\newtheorem{problem}{Problem}
\newtheorem{proposition}{Proposition}[section]
\newtheorem{lemma}{Lemma}[section]
\newtheorem{corollary}[theorem]{Corollary}
\newtheorem{example}{Example}[section]
\newtheorem{definition}[problem]{Definition}

\newcommand{\BEQA}{\begin{eqnarray}}
\newcommand{\EEQA}{\end{eqnarray}}
\newcommand{\define}{\stackrel{\triangle}{=}}
\bibliographystyle{IEEEtran}
\providecommand{\mbf}{\mathbf}
\providecommand{\pr}[1]{\ensuremath{\Pr\left(#1\right)}}
\providecommand{\qfunc}[1]{\ensuremath{Q\left(#1\right)}}
\providecommand{\sbrak}[1]{\ensuremath{{}\left[#1\right]}}
\providecommand{\lsbrak}[1]{\ensuremath{{}\left[#1\right.}}
\providecommand{\rsbrak}[1]{\ensuremath{{}\left.#1\right]}}
\providecommand{\brak}[1]{\ensuremath{\left(#1\right)}}
\providecommand{\lbrak}[1]{\ensuremath{\left(#1\right.}}
\providecommand{\rbrak}[1]{\ensuremath{\left.#1\right)}}
\providecommand{\cbrak}[1]{\ensuremath{\left\{#1\right\}}}
\providecommand{\lcbrak}[1]{\ensuremath{\left\{#1\right.}}
\providecommand{\rcbrak}[1]{\ensuremath{\left.#1\right\}}}
\theoremstyle{remark}
\newtheorem{rem}{Remark}
\newcommand{\sgn}{\mathop{\mathrm{sgn}}}
\providecommand{\abs}[1]{\left\vert#1\right\vert}
\providecommand{\res}[1]{\Res\displaylimits_{#1}} 
\providecommand{\norm}[1]{\left\lVert#1\right\rVert}
%\providecommand{\norm}[1]{\lVert#1\rVert}
\providecommand{\mtx}[1]{\mathbf{#1}}
\providecommand{\mean}[1]{E\left[ #1 \right]}
\providecommand{\fourier}{\overset{\mathcal{F}}{ \rightleftharpoons}}
%\providecommand{\hilbert}{\overset{\mathcal{H}}{ \rightleftharpoons}}
\providecommand{\system}{\overset{\mathcal{H}}{ \longleftrightarrow}}
    %\newcommand{\solution}[2]{\textbf{Solution:}{#1}}
\newcommand{\solution}{\noindent \textbf{Solution: }}
\newcommand{\cosec}{\,\text{cosec}\,}
\providecommand{\dec}[2]{\ensuremath{\overset{#1}{\underset{#2}{\gtrless}}}}
\newcommand{\myvec}[1]{\ensuremath{\begin{pmatrix}#1\end{pmatrix}}}
\newcommand{\mydet}[1]{\ensuremath{\begin{vmatrix}#1\end{vmatrix}}}
\numberwithin{equation}{subsection}
\makeatletter
\@addtoreset{figure}{problem}
\makeatother
\let\StandardTheFigure\thefigure
\let\vec\mathbf
\renewcommand{\thefigure}{\theproblem}
\def\putbox#1#2#3{\makebox[0in][l]{\makebox[#1][l]{}\raisebox{\baselineskip}[0in][0in]{\raisebox{#2}[0in][0in]{#3}}}}
     \def\rightbox#1{\makebox[0in][r]{#1}}
     \def\centbox#1{\makebox[0in]{#1}}
     \def\topbox#1{\raisebox{-\baselineskip}[0in][0in]{#1}}
     \def\midbox#1{\raisebox{-0.5\baselineskip}[0in][0in]{#1}}
\vspace{3cm}
\title{Assignment 1 (part2)}
\author{MUKUL KUMAR YADAV\\ EE20RESCH14003}
\maketitle
\newpage
\bigskip
\renewcommand{\thefigure}{\theenumi}
\renewcommand{\thetable}{\theenumi}
 Download Latex codes from here
\begin{lstlisting}
https://github.com/EE20RESCH14003/Assignment-1(part2)
\end{lstlisting}
%

%

%
\section{\textbf{ Matrix 3.9}}
\textbf{Question No. 73:} 

Find X so that $X\myvec{1&2&3\\1&4&5}=\myvec{-7&-8&-9\\2&4&6}$

\subsection{\textbf{Solution}}


    
Let $A=\myvec{1&2&3\\1&4&5}$ and $B=\myvec{-7&-8&-9\\2&4&6}$

Matrix A is 2x3 and B is 2x3 so matrix X must be 2x2


Assume matrix $X=\myvec{a&b\\c&d}$

\myvec{a&b\\c&d}\myvec{1&2&3\\1&4&5}=\myvec{-7&-8&-9\\2&4&6}

Multiplying the matrix X and A and comparing with matrix B

\begin{align}\label{eq1}
a+4b=-7\\ 
2a+5b=-8\\ 
3a+6b=-9  
\end{align}

\begin{align}\label{eq2}
c+4d=2\\
2c+5d=4\\
3c+6d=6
\end{align}
Solving set of equations for variables a and b

\myvec{
1&4&-7\\
2&5&-8\\
3&6&-9}
$\xleftrightarrow[R_3\rightarrow 3R_1-R_3]{R_2\rightarrow 2R_1-R_2}$
\myvec{
1&4&-7\\
0&3&-6\\
0&6&-12}

\myvec{
1&4&-7\\
0&3&-6\\
0&6&-12}
$\xleftrightarrow{R_3\rightarrow 2R_2-R_3}
\myvec{
1&4&-7\\
0&3&-6\\
0&0&0}$

The system is consistent and no free variables. Hence unique solution.

$3b=-6 \therefore b=-2$ and a=1

Similarly, for variables c and d 

\myvec{
1&4&2\\
2&5&4\\
3&6&6}
$\xleftrightarrow[R_3\rightarrow 3R_1-R_3]{R_2\rightarrow 2R_1-R_2}$
\myvec{
1&4&2\\
0&3&0\\
0&6&0}

\myvec{
1&4&2\\
0&3&0\\
0&6&0}
$\xleftrightarrow{R_3\rightarrow 2R_2-R_3}
\myvec{
1&4&2\\
0&3&0\\
0&0&0}$

System is consistent and no free variables. 
3d=0 $\therefore d=0$ and $c=2$

Hence, Matrix $X=\myvec{1&-2\\2&0}$


\end{document}